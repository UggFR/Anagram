\documentclass[a4paper]{article}
\usepackage[utf8x]{inputenc}
\usepackage[T1]{fontenc}
\usepackage{fullpage}
\usepackage[french]{babel}
\usepackage{listings}
\usepackage{parskip}
\usepackage{url}

\lstset{language=C++,basicstyle=\small\tt,showstringspaces=false}

\title{\vspace{-2cm}Base du Développement Logiciel}

\begin{document}
\maketitle

Une \emph{anagramme} est une permutation des lettres d'un mot pour
obtenir un nouveau mot. Par exemple, les mots \emph{onirique} et
\emph{ironique} ayant les mêmes lettres avec le même nombre d'occurrences
(un \emph{n}, deux \emph{i}, etc), ce sont des anagrammes l'un de
l'autre.

\section{Détection d'anagramme}

Définissez une fonction \lstinline|std::string normalize(std::string const &)|
ayant la spécification suivante : la chaîne renvoyée doit être la même
pour toutes les anagrammes du mot passé en argument ; elle doit être
différente pour deux mots qui ne sont pas des anagrammes.
Par exemple, \lstinline|"eiinoqru"| est une forme normale convenable pour
\emph{onirique} et \emph{ironique} (d'autres façons de normaliser sont
possibles). Indice: jeter un oeil à \texttt{std::sort}.\\

Utilisez cette fonction pour définir une fonction prenant deux chaînes de
caractères en argument et renvoyant \lstinline|true| si ces deux chaînes
sont anagrammes l'une de l'autre.\\

Testez votre fonction sur quelques exemples.

\section{Chargement de mots}

Définissez une fonction \lstinline|std::vector<std::string> load()| qui
lit le fichier \texttt{words.txt} fourni avec le sujet et renvoie un vecteur de chaînes de caractères contenant
un mot par case. Vous utiliserez préférentiellement \texttt{std::getline(std::istream\&, std::string \&)}
pour effectuer la lecture du fichier depuis un \texttt{std::ifstream}.\\

Testez votre fonction en affichant la longueur du vecteur (le nombre de
mots lus) ainsi que le dernier mot. (Note: le \emph{zizzyva} est un
coléoptère tropical.)

\section{Construction de dictionnaire}

L'objectif est maintenant de construire un dictionnaire permettant
d'inverser la fonction \lstinline|normalize|. Autrement dit, pour toute
chaîne \lstinline|s|, une expression ressemblant
à \lstinline|dict[normalize(s)]| doit permettre de retrouver la valeur
\lstinline|s|. Utilisez pour cela le conteneur
\lstinline|std::multimap| qui est une table de hachage à valeurs multiples:
\begin{lstlisting}
typedef std::multimap<std::string, std::string> dictionary;
\end{lstlisting}

Question : pourquoi est-ce que \lstinline|std::map| ne convient pas ?

Définissez une fonction
\lstinline|dictionary convert(std::vector<std::string> const &)| qui crée
et remplit un dictionnaire à l'aide des mots passés en argument.

Note : \lstinline|std::multimap::insert| prend en argument une
paire formée de la clé et de la valeur associée.

\section{Liste des anagrammes d'un mot}

Définissez une fonction qui renvoie toutes les anagrammes d'un mot (excepté
lui-même) présentes dans un dictionnaire :
\begin{lstlisting}
std::vector<std::string> anagrams(dictionary const &, std::string const &);
\end{lstlisting}

Note : la méthode \lstinline|equal_range| de
\lstinline|std::multimap| renvoie une paire d'itérateurs qui
délimitent la zone contenant toutes les entrées d'un dictionnaire ayant
la même clé.

Testez votre fonction en affichant les anagrammes des mots suivants (ou
un message d'erreur le cas échéant) :
\begin{lstlisting}
char const *test_words[] =
  { "anagram", "parrot", "abba", "insert", "silent" };
\end{lstlisting}

\end{document}
